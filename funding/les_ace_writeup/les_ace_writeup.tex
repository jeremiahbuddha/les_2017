\documentclass{tufte-handout}

\title{Leafleting Effectiveness Survey (LES)}
\author[Jack Norris \& Eric Roberts \& Jonathon Smith]{Jack Norris\thanks{Executive Director, Vegan Outreach} \& Eric Roberts\thanks{Research Manager, California Department of Health Services} \& Jonathon Smith\thanks{Technical Group Supervisor, Jet Propulsion Laboratory}}

\date{June 2017} % without \date command, current date is supplied
%\geometry{showframe} % display margins for debugging page layout

\usepackage{graphicx} % allow embedded images
  \setkeys{Gin}{width=\linewidth,totalheight=\textheight,keepaspectratio}
  \graphicspath{{graphics/}} % set of paths to search for images
\usepackage{amsmath}  % extended mathematics
\usepackage{booktabs} % book-quality tables
\usepackage{units}    % non-stacked fractions and better unit spacing
\usepackage{multicol} % multiple column layout facilities
\usepackage{colortbl}   % filler text
\usepackage{fancyvrb} % extended verbatim environments
\usepackage[export]{adjustbox}

  \fvset{fontsize=\normalsize}% default font size for fancy-verbatim environments

% Standardize command font styles and environments
\newcommand{\doccmd}[1]{\texttt{\textbackslash#1}}% command name -- adds backslash automatically
\newcommand{\docopt}[1]{\ensuremath{\langle}\textrm{\textit{#1}}\ensuremath{\rangle}}% optional command argument
\newcommand{\docarg}[1]{\textrm{\textit{#1}}}% (required) command argument
\newcommand{\docenv}[1]{\textsf{#1}}% environment name
\newcommand{\docpkg}[1]{\texttt{#1}}% package name
\newcommand{\doccls}[1]{\texttt{#1}}% document class name
\newcommand{\docclsopt}[1]{\texttt{#1}}% document class option name
\newenvironment{docspec}{\begin{quote}\noindent}{\end{quote}}% command specification environment

%\newcommand{\sitslong}[0]{\textsc{Spacecraft In The Shot}}% document class option name
\newcommand{\les}[0]{\textsc{LES}}% document class option name
\newcommand{\totalbooklets}[0]{\textsc{130000}}% document class option name                  
\newcommand{\stickercost}[0]{\textsc{22500}}% document class option name                  
\newcommand{\giftcardnum}[0]{\textsc{9000}}                                   
\newcommand{\giftcardcost}[0]{\textsc{67500}}

\begin{document}

\maketitle% this prints the handout title, author, and date

%\begin{itemize}
%\color{red}
%    \item Name of submitter and contact information (institutional affiliation, email address, phone number).
%    \item SMD research discipline with which the suggested new science investigation is best aligned (Astrophysics, Earth Science, Heliophysics, or Planetary	Science)
%    \item Desired orbit locations including any launch window constraints;
%\end{itemize}

% ---------------------------------------
% ---------------------------------------
\begin{abstract}
\noindent

Near the beginning of Fall semester 2017, a total of \totalbooklets~ 
students will be handed a leaflet. Half the leaflets will have a
pro animal rights message (\textbf{test books}), half will be on a
non-animal related topic (\textbf{control books}). Each booklet will have a
sticker advertising a ``\$5 Starbucks or Amazon Gift Card for taking
brief survey" (\textbf{Part 1 Survey}). Students answering the survey will be 
asked about their current consumption of animal products (\textbf{base rate}) 
and given a gift card. Two months later, just before Thanksgiving 
break, these same students will receive and email, "Thanks for taking 
our previous survey! We have a few more questions, of course for 
another \$5 giftcard!" (\textbf{Part 2 Survey}). Students will again
be asked about their current animal consumption. The data will
be examined to see if the pro-animal leaflets reduce the amount of
animal products consumed by the students (\textbf{post-test rate}), 
particularly to zero (the \textbf{``One Week Vegan"}). 

\end{abstract}

%\begin{marginfigure}[-2 in]%
%  \includegraphics[width=\textwidth]{sitspic3.png}
%  \caption{Illustration of \sitsshort~ footage of Europa spacecraft
%  during Europa close approach.}
%  \label{fig:sitsdepiction}
%\end{marginfigure}

%\begin{marginfigure}[0 in]%
%  \includegraphics[width=\textwidth]{sitsdepiction.png}\\
%  \includegraphics[width=.98\textwidth]{sitsdepiction2.png}
%  \caption{Illustration of a \sitsshort~ Imager capturing footage of 
%           Jupiter Orbit Insertion burn. After collection, the imager 
%           is left behind.}
%  \label{fig:sitsdepiction}
%\end{marginfigure}
  
% ---------------------------------------
% ---------------------------------------
%\section{Introduction}\label{sec:intro}

\newthought{The Leafleting Effectiveness Survery (\les)} will attempt
to quantify the impact of pro-animal leafleting on the later 
consumption of animal products. In particular it will look for 
``one week vegans", or leaflet recipients who report no consumption of animal 
products in the post-test survey week. 
The relationship between leafleting and one week vegans is of real
concern for the organization leading \les, Vegan Outreach (VO). VO's
primary intervention on behalf of animals is to hand out pro-vegetarian,
anti-speciest leaflets to college students. Verifying the impact of this
intervention has become an organizational priority at the highest level.
VO also understands that this is a general concern of the broader
animal protection community, and has sought the support and advice of
partner organizations to carry out \les.

In particular, VO seeking the participation of Animal Charity 
Evaluators (ACE) as a partner on \les. This document outlines VO's formal
request for a grant from ACE's Animal Advocacy Research Fund. VO is 
looking to ACE to finance (1) the printing of \totalbooklets~ 
survey-advertising stickers (\$\stickercost) (2) the purchase of
\giftcardnum~ gift cards to distribute as incentives for taking the 
survery (\$\giftcardcost), and (3) the creation of a public website
to share the results (\$\websitecost).

\begin{fullwidth}
{
\fontsize{9pt}{9pt}\selectfont
\vskip 1.5em
\noindent
\textsc{\sitsshort~ At A Glance}
\noindent
\begin{itemize}
    \item[] Technology development effort to enable the collection of 
            cinematic footage of robotic spacecraft operating at-location
            in deep space. Ultimate goal is to develop a self-contained 
            instrument capable of adoption into typical deep space mission payloads.
    \item[]
    \item[] The \sitsshort~ Imaging Instrument} (referred to as the \textit{instrument}) 
            is a self-contained unit that can be mounted 
            as spacecraft payload. It protects and deploys 
\end{itemize}
}
\end{fullwidth}

\clearpage
\sitslong~ has been sponsored the last 18 months by JPL's Office of 
Formulation (formerly the JPL Innovation Foundry). With their support, 
\sitsshort~ has received a NASA New Technology Report and provisional patent, passed through both 
A-Team and TeamXc study sessions, and emerged as a well developed concept 
with a point-engineering design.

% ---------------------------------------
% ---------------------------------------
\section{Concept of Operations}\label{sec:conops}

%\begin{itemize}
%\color{red}
%    \item 
%\end{itemize}

The \sitsshort~ system is designed to collect footage of its host spacecraft
operating at location. It does this by tasking an individual
\textit{imager} to fly a \texbf{media collection sortie}. These sorties are designed
in advance by analysts on the ground and uplinked to the \textit{instrument}. They identify
the time at which the \textit{imager} should be deployed and specify the trajectory and 
attitude profile that it needs to fly to collect the desired footage. There are 
three distinct phases involved in executing a sortie.

\begin{marginfigure}[-1.5 in]%
  \includegraphics[width=\textwidth]{sitscoop.png}
  \caption{\sitsshort~ \textit{imager} after deployment from the \textit{instrument} as it collects footage of the
  host spacecraft.}
  \label{fig:sitscoop}
\end{marginfigure}

\begin{fullwidth}
{  
\fontsize{9pt}{9pt}\selectfont
\vskip 1.5em
\noindent
\textsc{Media Collection Sortie}
\vskip 1em
\noindent
Phase 1: Deployment 
\begin{itemize}
  \item[] A sortie is initiated by the ejection of an \textit{imager} from its storage
  location inside the \textit{instrument}. The \textit{imager} will have been charged and tasked 
  with a trajectory prior to deployment. The ejection mechanism ensures that 
  \textit{imager} clears the immediate physical location of the host spacecraft, at which 
  point it brings the host spacecraft within
  its field of view and initiates a footage stream to 
  the \textit{instrument}. Once the feed has been registered, the \textit{imager} initiates its
  flight of the media collection trajectory.   
\end{itemize}
\noindent 
\vskip 1.5em
}
\end{fullwidth}

\begin{table*}
    \fontfamily{ppl}\selectfont
  \begin{tabular}{lll}
    \toprule
    Design Component & Maturity & Description \\
    \midrule
    Integrated Design & TRL 3 & TeamXc point design for \sitsshort~ system\\
    \midrule
    Hardware: & & \\
    Camera Stack & COTS & Cameras (Omnivision) and processors (Snapdragon) \\
    Local Communication & COTS & \textit{Instrument}-\textit{imager} communication via Qualcom Wifi Chips\\
    GNC & COTS & Sinclair reaction wheels, Epson IMU, DSSP thrusters \\
    Power & COTS & Panasonic Lithium-Ion Rechargeable Batteries \\
    \midrule
    Software: & & \\
    Monte & TRL 9 & JPL ground software for trajectory design and navigation \\   
    AutoNav & TRL 9 & JPL flight software for autonomous spacecraft navigation \\
    \bottomrule
  \end{tabular}
  \vskip 1.5em
  \caption{\sitsshort~ Technology Maturity Levels}
  \label{tab:dmat}
\end{table*}

% ---------------------------------------
\subsection{Collecting and Transmitting Footage}

Because the \textit{imagers} are discarded after a single use, emphasis has been placed on 
reducing both their physical size and cost so that more can be carried from 
mission start. Hardware is liberally sourced from the commercial cell 
phone market (Table \ref{tab:camstack}) which invests billions of dollars 
every year in just these initiatives (miniaturization and cost reduction). 

Each \textit{imager} is equipped with two cameras that independently stream
footage to the \textit{instrument}. Each camera has a dedicated Wifi chipset and
antenna to ensure a data-rate high enough to stream 4K video. On the receiving
end, the \textit{instrument} has four independent wifi receivers (chipset and antenna) which 
allows for the simultaneous operation of up to two \textit{imagers} (two dedicated 
channels for each \textit{imager}). The \textit{imagers} do not store any video; everything is 
streamed directly to the \textit{instrument} where it is logged in onboard storage and 
processed for downlink. As downlink becomes available, the \textit{instrument} queues 
up and sends data down through the main host spacecraft antenna.

\begin{table}
  \fontfamily{ppl}\selectfont
  \begin{tabular}{ll}
    \toprule
    Component & Description \\
    \midrule
    Camera &  Omnivision 13850 13.2 MP cell phone camera\\
    Processor & Snapdragon 820 Android processor \\
    Wifi Chipset & Qualcom AR9390 Chip 450Mbps \\ 
    Wifi Antenna & Johanson Technology 2450AT18B100 antenna\\
   \bottomrule
  \end{tabular}
  \caption{\sitsshort~ Camera Stack.}
  \label{tab:camstack}
\end{table}

\clearpage
{
\fontsize{9pt}{9pt}\selectfont
\vskip 1.5em
\noindent
\textsc{Camera Stack Design Summary}
\noindent
\begin{itemize}
    \item[] An S-Band (2.4 GHz) Wifi protocol is used for communication between the \textit{imagers} and \textit{instrument}.
    \item[] Each \textit{imager} has two independent cameras and streaming pipelines capable of
    processing 4K video (30 fps) or slow-motion 720P video (240 fps). 
    \begin{itemize}
      \item[] \textit{Imager} streaming pipeline consists of Omnivision Cameras (x2), 
      Snapdragon Processors (x2), Qualcomm Wifi Chips (x2), Johanson Antennas (x2)
    \end{itemize}
    \item[] The \textit{instrument} has four independent receivers capable of supporting 
    two simultaneously operating \textit{imagers}.
    \begin{itemize}
      \item[] \textit{Instrument} receiving pipeline consists of Qualcomm Wifi Chips (x4) and 
       Johanson Antennas (x4).
    \end{itemize}
    \item[] Footage is saved on board the \textit{instrument} and preprocessed for 
    eventual downlink to Earth via the host spacecraft communication system.
\end{itemize}
\vskip 1.5em
}


% ---------------------------------------
\subsection{Autonomous operation}

JPL has an astrodynamic software toolkit called Monte\footnote{montepy.jpl.nasa.gov} 
that is used for high-accuracy trajectory design and navigation. The spacecraft 
trajectories produced by Monte are routinely used by science instrument operators to design 
their observation profiles. Monte can be configured to do an analogous task for
\sitsshort, which is to help operators design \textit{imager} sortie trajectories that 
capture dynamic cinematic shots of the spacecraft and its surrounding environment.
These trajectories will be created well in advance of the actual sortie, and 
uplinked for the \textit{imagers} to fly autonomously. 

After an \textit{imager} has been ejected, it begins flying the sortie only after it
has cleared a minimum-operational distance from the host spacecraft.\footnote{The current
design has this set at 3m, but it can be adjusted if the host spacecraft
has extended features (solar panels, magnetic booms) that need to be 
avoided.} At this point, the \textit{imager} uses its onboard IMU to dead-recon flight
of the reference trajectory. JPL's AutoNav flight software is used to adjust
the pointing of the \textit{imager} cameras in real time, to ensure the integrity of
the shot even if the \textit{imager} departs from its nominal flight plan.\footnote{AutoNav has 
flight heritage on several NASA missions, including Deep Impact and Stardust missions, where it was used with great 
success.} 

After the sortie is complete, the \textit{instrument} will autonomously process the collected
footage and segment it for downlink. Initially, only a low-resolution version of the
footage will be returned, allowing operators to prioritize in what order the 
full-resolution segments will be returned.

\clearpage
% ---------------------------------------
% ---------------------------------------
\section{Path to Flight}

%\begin{itemize}
%\color{red}
%    \item Any preliminary studies that should be undertaken in advance of a 
%    decision to proceed in order to determine the feasibility, sensitivity, 
%    cost, or other characteristics of the new science investigation.	
%    \item List of payload services required from integration to operations, including 
%    but not limited to cleanliness level, unique test and checkout, temperature, access, 
%    power and environment.
%    \item Identification of any hazardous, explosive, chemical, or biological aspects of the
%    payload.
%\end{itemize}

In order for \sitsshort~ to be ready for flight on-board NSTP-Sat in 2021, an official,
funded \sitslong~ project needs to be established in FY18. The first order of business 
is to alter the system design to fit the NSTP-Sat paradigm. The current design is 
a merged version of the \sitsshort~ concept into a 12U CubeSat (6U for \sitsshort~ 
and 6U for the CubeSat bus) appropriate for flight as a standalone NASA Technology 
Demonstration Mission. Adapted to NSTP-Sat, \sitsshort~ would drop off the 6U bus and 
assume its proper form as a true payload instrument. During this redesign, the 
\textit{instrument} can be custom-fitted to NSTP-Sat's size, weight, and power 
requirements.\footnote{The overall size and weight of the resulting \textit{instrument} can
be adjusted by adding or removing \textit{imagers} from its manifest.} 

\begin{fullwidth}
{
\fontsize{9pt}{9pt}\selectfont
\vskip 1.5em
\noindent
\textsc{Timeline for Developing Flight Instrument}
\noindent
\begin{itemize}
    \item[\textbf{ATP}] For project start beginning FY18.
    \item[\textbf{ATP + 3 months}] Complete engineering redesign.
    \item[\textbf{ATP + 4 months}] SRR.
    \item[\textbf{ATP + 9 months}] Finish full redesign, begin hardware prototype.
    \item[\textbf{ATP + 11 months}] PDR.
    \item[\textbf{ATP + 16 months}] Refine design based on prototype testing. Begin work on flight
     hardware. 
    \item[\textbf{ATP + 19 months}] CDR.
     
    \item[\textbf{ATP + 26 months}] Complete flight hardware, begin internal testing. 
    \item[\textbf{ATP + 29 months}] Deliver to NSTP-Sat Payload Integration Center.
    \item[\textbf{ATP + 31 months}] Start Payload Integration.
\end{itemize}
}
\end{fullwidth}

Once \sitsshort~~ has been redesigned, an accelerated hardware development program
needs to be initiated. Because so much of the system is built on high-TRL or COTS 
technology, rapid fabrication of the \textit{instrument} and \textit{imagers} is feasible in 
a very short timeframe. During the prototyping phase, manufacturers will be 
engaged to help reconfigure their products to further miniaturize the \textit{imager} 
design, which will in turn reduce the size of the overall instrument package.
Additionally, the non-space hardened components will be vacuum and radiation tested 
and any which show signs of failing before the planned \textit{imager} lifetime of 15 minutes
can be replaced. Lessons learned from the prototype will be wrapped into the final design
and fabrication of a flight-ready instrument.

% ---------------------------------------
% ---------------------------------------
%\section{Closing Thoughts}\label{sec:close}





\end{document}  
